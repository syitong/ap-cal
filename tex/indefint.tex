\section[IndefInt]{Indefinite Integral}
\begin{frame}{Antiderivative/indefinite integral}


\[
\int f\left(x\right)\,\mathrm{d}x
\]



\pause{}
\begin{itemize}
\item antiderivative/indefinite integral of $f\left(x\right)$ is the function
whose derivative is $f\left(x\right)$.
\item For each function $f\left(x\right)$, its antiderivative is not unique.
\alert{Always remember to add $C$}.
\end{itemize}
\end{frame}

\begin{frame}{Basic formulas}


Remember all the formulas on p217 except for 13, 14, 17, 18, 19
\begin{example}
\[
\int\left(x^{4}+\sqrt[3]{x^{2}}-\frac{2}{x^{2}}-\frac{1}{3\sqrt[3]{x}}\right)\,\mathrm{d}x
\]

\end{example}

\end{frame}

\begin{frame}{Substitution rule}

\begin{onlyenv}<1>


If you have an integral of the form
\[
\int F\left(f\left(x\right)\right)f'\left(x\right)\,\mathrm{d}x
\]
then it equals
\[
\int F\left(u\right)\,\mathrm{d}u
\]
where $u=f\left(x\right)$.


Or you can consider that whenever a function passes through the differentiation
sign $\mathrm{d}$ from left to right, it becomes its antiderivative,
and when it passes from right to left, it becomes its derivative.
And everything after $\mathrm{d}$ can be viewed as a whole part.

\end{onlyenv}



\begin{onlyenv}<2>

\begin{example}
{\footnotesize{}
\[
\int\left(1-x^{2}\right)x\,\mathrm{d}x
\]
}{\footnotesize \par}
\end{example}


\pause{}
\begin{sol}
{\tiny{}
\begin{eqnarray*}
\int\left(1-x^{2}\right)x\,\mathrm{d}x & = & -\frac{1}{2}\int\left(1-x^{2}\right)\left(-2x\right)\,\mathrm{d}x\quad\boxed{u=1-x^{2}\,,u'=-2x}\\
 & = & -\frac{1}{2}\int u\cdot u'\,\mathrm{d}x\\
 & = & -\frac{1}{2}\int u\,\mathrm{d}u\\
 & = & -\frac{1}{2}\cdot\frac{1}{2}u^{2}\\
 & = & -\frac{1}{4}\left(1-x^{2}\right)^{2}\,.
\end{eqnarray*}
}{\tiny \par}
\end{sol}
\end{onlyenv}



\begin{onlyenv}<3>

\begin{example}
\[
\int\left(2x^{3}-1\right)^{5}x^{2}\,\mathrm{d}x\,,\int\sqrt[3]{1-x}\,\mathrm{d}x\,,\int\frac{e^{x}}{1-2e^{x}}\,\mathrm{d}x
\]
\[
\int\sin\left(1-2y\right)\,\mathrm{d}y\,,\int e^{\tan y}\sec^{2}y\,\mathrm{d}y\,,\int\frac{\cos\sqrt{x}}{\sqrt{x}}\,\mathrm{d}x
\]

\end{example}

\end{onlyenv}

\end{frame}

\begin{frame}{Integration by partial fractions}

\begin{onlyenv}<1>

\begin{enumerate}
\item Polynomial division;
\item Denominator factorization;
\item Determine coefficients;
\item Use \alert{substitution} and rule for $\frac{1}{x}$ to integrate.
\end{enumerate}
\end{onlyenv}



\begin{onlyenv}<2>

\begin{example}
\[
\int\frac{x^{2}-x+4}{x^{3}-3x^{2}+2x}\,\mathrm{d}x
\]

\end{example}

\end{onlyenv}

\end{frame}

\begin{frame}{Integration by parts}


\begin{eqnarray*}
\int f\left(x\right)g'\left(x\right)\,\mathrm{d}x & = & \int f\left(x\right)\,\mathrm{d}g\left(x\right)\\
 & = & f\left(x\right)g\left(x\right)-\int g\left(x\right)\,\mathrm{d}f\left(x\right)\\
 & = & f\left(x\right)g\left(x\right)-\int g\left(x\right)f'\left(x\right)\,\mathrm{d}x\,.
\end{eqnarray*}



\pause{}
\begin{example}
\[
\int x\cos x\,\mathrm{d}x\,,\int e^{x}\cos x\,,\int x^{4}\ln x\,\mathrm{d}x
\]

\end{example}


\pause{}


What can you conclude?

\end{frame}

\begin{frame}{Application of indefinite integral}

\begin{onlyenv}<1-2>

\begin{example}
The velocity of a particle moving along a line is given by $v\left(t\right)=4t^{3}-3t^{2}$
at time $t$. If the particle is initially at $x=3$ on the line,
find its position when $t=2$.
\end{example}


\pause{}
\begin{sol}
\[
s\left(t\right)=\int v\left(t\right)\,\mathrm{d}t
\]
\[
s\left(0\right)=3
\]

\end{sol}
\end{onlyenv}



\begin{onlyenv}<3>

\begin{example}
Suppose that $a\left(t\right)$, the acceleration of a particle at
time $t$, is given by $a\left(t\right)=4t-3$, that $v\left(1\right)=6$,
and that $f\left(2\right)=5$, where $f\left(t\right)$ is the position
function.
\begin{enumerate}
\item Find $v\left(t\right)$ and $f\left(t\right)$;
\item Find the position of the particle when $t=1$.
\end{enumerate}

\end{example}

\end{onlyenv}

\end{frame}



