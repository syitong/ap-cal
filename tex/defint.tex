\section[DefInt]{Definite Integrals}
\begin{frame}{Properties of definite integrals}

\begin{itemize}
\item 
\[
\int_{a}^{c}f\left(x\right)\,\mathrm{d}x+\int_{c}^{b}f\left(x\right)\,\mathrm{d}x=\int_{a}^{b}f\left(x\right)\,\mathrm{d}x
\]

\end{itemize}

\pause{}
\begin{itemize}
\item 
\[
\int_{a}^{b}f\left(x\right)\,\mathrm{d}x=-\int_{b}^{a}f\left(x\right)\,\mathrm{d}x
\]

\end{itemize}

\pause{}
\begin{itemize}
\item If $f\left(x\right)\le g\left(x\right)$ for all $x\in\left[a,b\right]$,
then
\[
\int_{a}^{b}f\left(x\right)\,\mathrm{d}x\le\int_{a}^{b}g\left(x\right)\,\mathrm{d}x\,.
\]

\end{itemize}
\end{frame}

\begin{frame}{Fundamental theorem of calculus}


\[
\int_{a}^{b}f\left(x\right)\,\mathrm{d}x=F\left(b\right)-F\left(a\right)
\]
where $F$ is the antiderivative of $f$. Or
\[
\frac{\mathrm{d}}{\mathrm{d}x}\int_{a}^{x}f\left(t\right)\,\mathrm{d}t=f\left(x\right)\,.
\]



\pause{}


Verbally, FTC says that integration and differentiation are two reverse
processes for a function.

\end{frame}

\begin{frame}{Evaluate definite integrals}

\begin{onlyenv}<1-2>

\begin{example}
\[
\int_{-1}^{2}\left(3x^{2}-2x\right)\,\mathrm{d}x
\]
\end{example}

\begin{uncoverenv}<2>

\begin{sol}
\begin{eqnarray*}
\int_{-1}^{2}\left(3x^{2}-2x\right)\,\mathrm{d}x & = & \left(x^{3}-x^{2}\right)\big|_{-1}^{2}\\
 & = & 6\,.
\end{eqnarray*}

\end{sol}
\end{uncoverenv}

\end{onlyenv}



\begin{onlyenv}<3-5>

\begin{example}
\[
\int_{5}^{8}\frac{\mathrm{d}y}{\sqrt{9-y}}
\]
\end{example}

\begin{onlyenv}<4>

\begin{sol}
{\scriptsize{}
\begin{eqnarray*}
\int_{5}^{8}\frac{\mathrm{d}y}{\sqrt{9-y}} & = & -\int_{5}^{8}\left(9-y\right)^{-\frac{1}{2}}\left(-1\right)\,\mathrm{d}y\\
 & = & -\int_{5}^{8}\left(9-y\right)^{-\frac{1}{2}}\,\mathrm{d}\left(9-y\right)\\
 & = & -2\left(9-y\right)^{\frac{1}{2}}\bigg|_{5}^{8}\\
 & = & 2
\end{eqnarray*}
}{\scriptsize \par}
\end{sol}
\end{onlyenv}



\begin{onlyenv}<5>

\begin{sol}
{\scriptsize{}
\begin{eqnarray*}
\int_{5}^{8}\frac{\mathrm{d}y}{\sqrt{9-y}} & = & -\int_{5}^{8}\left(9-y\right)^{-\frac{1}{2}}\left(-1\right)\,\mathrm{d}y\\
 & = & -\int_{5}^{8}\left(9-y\right)^{-\frac{1}{2}}\,\mathrm{d}\left(9-y\right)\\
 & = & -\int_{9-5}^{9-8}u^{-\frac{1}{2}}\,\mathrm{d}u\\
 & = & -2u^{\frac{1}{2}}\bigg|_{4}^{1}
\end{eqnarray*}
}{\scriptsize \par}
\end{sol}
\end{onlyenv}

\end{onlyenv}



\begin{onlyenv}<6-7>

\begin{example}
\[
\int_{0}^{1}e^{x}x\,\mathrm{d}x
\]
\end{example}

\begin{uncoverenv}<7>

\begin{sol}
{\scriptsize{}
\begin{eqnarray*}
\int_{0}^{1}e^{x}x\,\mathrm{d}x & = & \int_{0}^{1}x\,\mathrm{d}e^{x}\\
 & = & xe^{x}\bigg|_{0}^{1}-\int_{0}^{1}e^{x}\,\mathrm{d}x\\
 & = & e-0-\left(e-1\right)\\
 & = & 1
\end{eqnarray*}
}{\scriptsize \par}
\end{sol}
\end{uncoverenv}

\end{onlyenv}

\end{frame}

\begin{frame}{Represent functions using definite integral}

\begin{example}
\[
\frac{\mathrm{d}}{\mathrm{d}x}\int_{-1}^{x}\sqrt{1+\sin^{2}t}\,\mathrm{d}t\,,\frac{\mathrm{d}}{\mathrm{d}x}\int_{x}^{1}e^{-t^{2}}\,\mathrm{d}t
\]
\[
\frac{\mathrm{d}}{\mathrm{d}x}\int_{1}^{x}\frac{\mathrm{d}t}{3+t}
\]

\end{example}



\begin{example}
Find 
\[
\lim_{h\to0}\frac{1}{h}\int_{x}^{x+h}\sqrt{e^{t}-1}\,\mathrm{d}t\,.
\]

\end{example}

\end{frame}

\begin{frame}{Integrals involving parametrically defined functions}

\begin{example}
Evaluate $\int_{-2}^{2}y\,\mathrm{d}x$, where $x=2\sin\theta$, $y=3\cos\theta$.
\end{example}


\pause{}
\begin{example}
Sketch the curve given by $x=t-\sin t$, $y=1-\cos t$. Find out the
area between the curve and $x$-axis for one period.
\end{example}

\end{frame}

\begin{frame}{Riemann sum and approximation}

\begin{onlyenv}<1>


\[
\lim_{n\to\infty}\sum_{k=1}^{n}f\left(x_{k}\right)\Delta x_{k}=\int_{a}^{b}f\left(x\right)\,\mathrm{d}x\,.
\]
\[
\sum_{k=1}^{n}f\left(x_{k}\right)\Delta x_{k}\approx\int_{a}^{b}f\left(x\right)\,\mathrm{d}x\,.
\]


\end{onlyenv}



\begin{onlyenv}<2>

\begin{example}
Approximate $\int_{0}^{2}x^{3}\,\mathrm{d}x$ by using four subintervals
of equal width and calculating:
\begin{enumerate}
\item the left sum,
\item the right sum,
\item the midpoint sum,
\item the trapezoidal sum,
\item the integral.
\end{enumerate}

\end{example}

\end{onlyenv}

\end{frame}

\begin{frame}{Average value of a function}


\[
\frac{1}{b-a}\int_{a}^{b}f\left(x\right)\,\mathrm{d}x
\]
The average value of a \alert{continuous} function equals the value
of the function at some point $c$ within the interval $\left(a,b\right)$.
\begin{example}
Find the average value of $y=\ln x$ on the interval $\left[1,4\right]$.
\end{example}

\end{frame}

\begin{frame}{Application of these techniques}

\begin{example}
p264 Example 28.
\end{example}



\begin{example}
p266 Example 29.
\end{example}



\begin{example}
p275 Example 41.
\end{example}

\end{frame}



