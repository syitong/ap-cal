\section{Differentiation}
\begin{frame}{Definition of derivative}


Derivative is defined as the limit of difference quotient,
\[
\frac{\mathrm{d}}{\mathrm{d}x}f\left(x\right)=f'\left(x\right)=\lim_{\Delta x\to0}\frac{f\left(x+\Delta x\right)-f\left(x\right)}{\Delta x}\,.
\]

\begin{onlyenv}<2-4>

\begin{itemize}
\item The difference quotient is the slope of \uncover<3-4>{secant line}.
\item Its limit, the derivative, is the slope of \uncover<4>{tangent line}.
\end{itemize}
\end{onlyenv}



\begin{onlyenv}<5-7>

\begin{itemize}
\item The difference quotient is the \uncover<6-7>{average} rate of change.
\item Its limit, the derivative, is the \uncover<7>{instantaneous} rate
of change.
\end{itemize}
\end{onlyenv}

\end{frame}

\begin{frame}{Derivatives of basic functions}

\begin{onlyenv}<1>


Derivative of $\tan(x)$, $a^{x}$, $\arcsin(x)$, $\arccos(x)$,
$\arctan(x)$.

\end{onlyenv}



\begin{onlyenv}<2>

\begin{example}
If $y=\left(x^{2}+x\right)\cdot\sin x$, find $y'\left(\frac{\pi}{2}\right)$
and $y''\left(\frac{\pi}{2}\right)$.
\end{example}



\begin{example}
If $f\left(v\right)=\frac{2v}{1-2v^{2}}$, find $f'\left(v\right)$.
\end{example}

\end{onlyenv}

\end{frame}

\begin{frame}{Chain rule}

\begin{example}
If $f\left(x\right)=\left(3x+2\right)^{5}$, then find $f'\left(x\right)$.
\end{example}



\begin{example}
If $f\left(x\right)=\frac{5}{\sqrt{\left(1-x^{2}\right)^{3}}}$, find
$f'\left(x\right)$.
\end{example}



\begin{example}
p181-75
\end{example}

\end{frame}

\begin{frame}{Differentiability and continuity}


Differentiability implies continuity,


but continuity does not imply differentiability.


Think about the graph of $\left|x\right|$ and $x^{1/3}$.


\pause{}
\begin{example}
p182-10
\end{example}

\end{frame}

\begin{frame}{Estimating a derivative}

\begin{example}
Numerically, p153-39,40
\end{example}


\pause{}
\begin{example}
Graphically, p142-24, p180-68
\end{example}

\end{frame}

\begin{frame}{Derivatives of parametric curves}

\begin{onlyenv}<1>


\[
\frac{\mathrm{dy}}{\mathrm{d}x}=\frac{\frac{\mathrm{d}y}{\mathrm{d}t}}{\frac{\mathrm{d}x}{\mathrm{d}t}}
\]
\[
\frac{\mathrm{d}^{2}y}{\mathrm{d}x^{2}}=\frac{\frac{\mathrm{d}}{\mathrm{d}t}\left(\frac{\mathrm{d}y}{\mathrm{d}x}\right)}{\frac{\mathrm{d}x}{\mathrm{d}t}}
\]


\end{onlyenv}



\begin{onlyenv}<2>

\begin{example}
Find the equation of the tangent to the curve given by
\[
\begin{cases}
x=2\sin\theta\\
y=\cos2\theta
\end{cases}
\]
for $\theta=\frac{\pi}{6}$.


\end{example}

\end{onlyenv}

\end{frame}

\begin{frame}{Implicit differentiation}


\alert{Just remember, every thing in the equation is a function of independent
variable. That means you need apply Product, Quotient and Chain rule
to them.}
\begin{example}
Find the derivative $y'$ and $y''$ for
\[
x^{2}-2xy+3y^{2}=2
\]
and
\[
x\sin y=\cos\left(x+y\right)
\]



\end{example}

\end{frame}

\begin{frame}{Derivative of the inverse of a function}

\begin{onlyenv}<1>


If a function $y=f\left(x\right)$ has its inverse $x=f^{-1}\left(y\right)$.
Then
\begin{eqnarray*}
f^{-1}\left(f\left(x\right)\right) & = & x\\
\frac{\mathrm{d}}{\mathrm{d}y}f^{-1}\left(y\right)\bigg|_{y=f\left(x\right)}\cdot\frac{\mathrm{d}}{\mathrm{d}x}f\left(x\right) & = & 1\\
\frac{\mathrm{d}}{\mathrm{d}y}f^{-1}\left(y\right)\bigg|_{y=f\left(x\right)} & = & \frac{1}{\frac{\mathrm{d}}{\mathrm{d}x}f\left(x\right)}\,.
\end{eqnarray*}


\end{onlyenv}



\begin{onlyenv}<2,3>

\begin{example}
If $f\left(3\right)=8$ and $f'\left(3\right)=5$, what do we know
about $f^{-1}$?\end{example}

\begin{uncoverenv}<3>

\begin{sol}
First, $f^{-1}\left(8\right)=3$.

\[
\frac{\mathrm{d}}{\mathrm{d}y}f^{-1}\bigg|_{y=8}=\frac{1}{\frac{\mathrm{d}}{\mathrm{d}x}f\bigg|_{x=3}}=\frac{1}{5}\,.
\]

\end{sol}
\end{uncoverenv}

\end{onlyenv}



\begin{onlyenv}<4>

\begin{example}
Let $y=f\left(x\right)=x^{3}+x-2$, and let $g$ be the inverse function.
Evaluate $g'\left(0\right)$.
\end{example}

\end{onlyenv}

\end{frame}

\begin{frame}{The mean value theorem (differentiation)}

\begin{onlyenv}<1>


\[
\frac{f\left(b\right)-f\left(a\right)}{b-a}=f'\left(c\right)\mbox{ for some }c\in\left(a,b\right)\,.
\]
Verbally, the \alert{average rate of change} of some function $f$
on some interval 


equals the \alert{instantaneous rate of change} at some point in
that interval.

\end{onlyenv}



\begin{onlyenv}<2>

\begin{example}
You left home one morning and drove to a cousin's house $300$ miles
away, arriving $6$ hours later. What does the mean value theorem
say about your speed along the way?
\end{example}

\end{onlyenv}

\end{frame}

\begin{frame}{l'Hôspital's rule}

\begin{onlyenv}<1>

\begin{enumerate}
\item First, the expression must be an indeterminate form.
\item Find derivatives of the denominator and numerator.
\item Compute the limit of quotient of derivatives.

\begin{enumerate}
\item If the limit exists or equals $\infty$, then it is the value of the
original limit;
\item if the limit does not exist, then we cannot get a conclusion;
\item if the it is still an indeterminate form, repeat step 2.
\end{enumerate}
\end{enumerate}
\end{onlyenv}



\begin{onlyenv}<2>

\begin{example}
$\lim_{x\to-2}\frac{x^{3}+8}{x^{2}-4}$
\end{example}



\begin{example}
$\lim_{x\to2}\frac{x^{3}+8}{x^{2}+4}$
\end{example}



\begin{example}
$\lim_{x\to\infty}x\sin\frac{1}{x}$
\end{example}

\end{onlyenv}

\end{frame}

\begin{frame}{Recognize a given limit as a derivative}

\begin{example}
$\lim_{h\to0}\frac{\left(2+h\right)^{4}-2^{4}}{h}$
\end{example}



\begin{example}
$\lim_{h\to0}\frac{1}{h}\left(\frac{1}{2+h}-\frac{1}{2}\right)$
\end{example}

\end{frame}



