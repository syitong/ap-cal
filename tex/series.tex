\section{Sequences and Series}
\begin{frame}{Sequences and series}
    \[
    \lim_{n\to\infty}\frac{1}{n}\quad\sum_{n=1}^{\infty}\frac{1}{n}
    \]
    \pause{}

    \[
    \lim_{n\to\infty}\frac{1}{2^{n}}\quad\sum_{n=1}^{\infty}\frac{1}{2^{n}}
    \]
    \pause{}
    
    \[
    \lim_{n\to\infty}\frac{3n^{4}+5}{4n^{4}-7n^{2}+9}
    \]
\end{frame}

\begin{frame}{Convergence of series}
    The definition of sum of infinite many terms:
    \[
    \sum_{n=1}^{\infty}a_{n}=\lim_{N\to\infty}\sum_{n=1}^{N}a_{n}\,.
    \]
    When the limit exists, it is called convergent.
    
    
    For a series to be convergent, its ``tail'' must shrink very quickly
    \[
    \lim_{n\to\infty}a_{n}=0\,.
    \]
    When $\sum\left|a_{n}\right|$ is convergent, we call the series $\sum a_{n}$
    \alert{absolutely convergent}. What is \alert{conditional convergence}?

\end{frame}

\begin{frame}{Geometric series}


\begin{eqnarray*}
\sum_{n=1}^{\infty}ar^{n-1} & = & \lim_{N\to\infty}\frac{a\left(1-r^{N}\right)}{1-r}\\
 & = & \frac{a}{1-r}\,,
\end{eqnarray*}
when \uncover<2>{$\left|r\right|<1$.}

\pause{}


So no matter $r$ is postive or negative, if a geometric series converges,
it converges absolutely.

\end{frame}

\begin{frame}{$p$-series}


\[
\sum_{n=1}^{\infty}\frac{1}{n^{p}}
\]

\begin{itemize}
\item It is convergent when $p>1$;
\item It is divergent when $p\le1$.
\item $\sum_{n=1}^{\infty}\frac{\left(-1\right)^{n}}{n^{p}}$ is always
convergent, but not always absolutely. Why?
\end{itemize}
\end{frame}

\begin{frame}{Telescopic series}


\begin{eqnarray*}
\sum_{n=1}^{\infty}\frac{1}{n\left(n+1\right)} & = & \lim_{N\to\infty}\sum_{n=1}^{N}\left(\frac{1}{n}-\frac{1}{n+1}\right)\\
 & = & \lim_{N\to\infty}\left(1-\frac{1}{N+1}\right)\\
 & = & 1\,.
\end{eqnarray*}

\begin{example}
\[
\sum_{n=1}^{\infty}\frac{1}{n\left(n+2\right)}
\]

\end{example}

\end{frame}

\begin{frame}{Tests for convergence}

\begin{onlyenv}<1>

\begin{itemize}
\item Ratio test.
\[
\lim_{n\to\infty}\left|\frac{a_{n+1}}{a_{n}}\right|=L
\]

\item Comparison test. Only for positive series. Want to show convergence,
amplify it; want to show divergence, shrink it.
\item Integral test. Only for positive series. The integral converges $\int_{1}^{\infty}f\left(n\right)\,\mathrm{d}n$
iff the series $\sum f\left(n\right)$ converges.
\item Alternating series test.
\[
\left|a_{n+1}\right|\le\left|a_{n}\right|\mbox{ and }\lim_{n\to\infty}a_{n}=0\,.
\]

\end{itemize}
\end{onlyenv}



\begin{onlyenv}<2>

\begin{example}
\[
\sum_{n=1}^{\infty}\frac{1}{1+n^{4}}\,,\sum_{n=1}^{\infty}\left(\frac{n}{2n+1}\right)^{n}\,,\sum_{n=1}^{\infty}\frac{\sin\frac{n\pi}{3}}{n^{2}}
\]

\end{example}



\begin{example}
\[
\sum_{n=1}^{\infty}\frac{\left(-1\right)^{n+1}}{\sqrt[3]{n+1}}
\]

\end{example}

\end{onlyenv}

\end{frame}

\begin{frame}{Power series}


\[
\sum_{k=0}^{\infty}c_{k}x^{k}\quad\sum_{k=0}^{\infty}c_{k}\left(x-a\right)^{k}
\]
Use ratio test to control radius of convergence.



\begin{example}
For what $x$ does
\[
\sum_{n=1}^{\infty}\frac{\left(-1\right)^{n-1}x^{n-1}}{n+1}
\]
converge?
\end{example}

\end{frame}

\begin{frame}{Taylor series}
Some (\alert{analytic}) functions can be expressed as power series for all real numbers
or within some interval.
\[
f\left(x\right)=\sum_{n=0}^{\infty}\frac{f^{\left(n\right)}\left(a\right)}{n!}\left(x-a\right)^{n}\,\mbox{for }\left|x-a\right|<R\,.
\]

\begin{example}
p667-50
\end{example}

\end{frame}

\begin{frame}{Counterexample}
        \only<1>{
        Note that it is true that with some regularity assumptions
        \[f(x)=\sum_{k=0}^{n}\frac{f^{(k)}(a)}{k!}\left(x-a\right)^k+\frac{f^{(k+1)}(\xi)}{(k+1)!}\left(x-a\right)^{(k+1)}\,.\]
        But it is not true that
        \[f(x)=\sum_{k=0}^{\infty}\frac{f^{(k)}(a)}{k!}\left(x-a\right)^k\,,\]
        even when the series converges in $(a-R,a+R)$ for some $R>0$.
}
        \only<2>{
        See the following example (by Cauchy)
        \[f(x)=
        \begin{cases}
                e^{-1/x^2} & x\ne 0\,;\\
                0 & x=0\,.
        \end{cases}\]
        The remainder formula is still correct but the derivatives oscillate severely.
}
\end{frame}

\begin{frame}{Basic series}


$\sin x$, $\cos x$, $e^{x}$, $\ln\left(1+x\right)$, $\frac{1}{1-x}$

\end{frame}

\begin{frame}{Calculus on series}

\begin{onlyenv}<1>
Within the interval of convergence, series can be added, multiplied
by a constant, integrated and differentiated \alert{term by term}.
For example,
\[
\arctan x=\int_{0}^{x}\frac{\mathrm{d}t}{1+t^{2}}=\sum_{n=0}^{\infty}\int\left(-1\right)^{n}t^{2n}\,\mathrm{d}t\,.
\]
\end{onlyenv}

\begin{onlyenv}<2>

\begin{example}
    p687-6(d), 10, 20
\end{example}

\end{onlyenv}

\end{frame}

\begin{frame}{Approximate functions}

\begin{onlyenv}<1>
\[
f\left(x\right)\approx\sum_{n=1}^{N}\frac{f^{\left(n\right)}\left(a\right)}{n!}\left(x-a\right)^{n}
\]
with the error
\[
\left|f\left(x\right)-\sum_{n=1}^{N}\frac{f^{\left(n\right)}\left(a\right)}{n!}\left(x-a\right)^{n}\right|\le\left|\frac{{\displaystyle \max_{\left|t-a\right|\le R}}f^{\left(n+1\right)}\left(t\right)}{\left(n+1\right)!}\left(x-a\right)^{n+1}\right|
\]
for $\left|x-a\right|\le R$.

If it is an alternating series with $\left|a_{n+1}\right|<\left|a_{n}\right|$,
then
\[
\left|f\left(x\right)-\sum_{n=1}^{N}\frac{f^{\left(n\right)}\left(a\right)}{n!}\left(x-a\right)^{n}\right|\le\left|\frac{f^{\left(n+1\right)}\left(a\right)}{\left(n+1\right)!}\left(x-a\right)^{n+1}\right|\,.
\]
\end{onlyenv}

\begin{onlyenv}<2>

\begin{example}
For what values of $x$ is the approximate formula
\[
\ln\left(1+x\right)\approx x-\frac{x^{2}}{2}
\]
accurate to $0.001$?\end{example}

\end{onlyenv}

\end{frame}


