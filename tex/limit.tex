\section[Limits]{Limits and Continuity}
\begin{frame}{Basic limits}


\[
\lim_{x\to\infty}\frac{1}{x}=\quad,\lim_{x\to0-}\frac{1}{x}=\quad,\lim_{x\to0}\frac{\sin x}{x}=\quad
\]
\[
\lim_{x\to0+}\left(1+x\right)^{1/x}=\quad,\lim_{x\to\infty}\arctan x=\quad\lim_{x\to\infty}e^{-x}=
\]

\begin{example}
p128-4
\end{example}

\end{frame}

\begin{frame}{Limits laws}


\[
\lim_{x\to0+}\left[\left(1+x\right)^{1/x}+\frac{\sin x}{x}\right]=
\]
\[
\lim_{x\to\infty}\left(\frac{2}{x}-\frac{1}{x^{2}}\right)=
\]
\[
\lim_{x\to0}\frac{x}{x}=
\]


\end{frame}

\begin{frame}{Limits of continuous functions}

\begin{onlyenv}<1>


If $f$ is continuous,
\[
\lim_{x\to2}f\left(x\right)=
\]
\[
\lim_{x\to\infty}f\left(\frac{1}{x}\right)=
\]


\end{onlyenv}



\begin{onlyenv}<2>

\begin{example}
p128-1(a),(h),
\[
\lim_{x\to\infty}f\left(\frac{3x}{x-1}\right)\,,\lim_{x\to0}f\left(\frac{3x}{x}\right)\,.
\]

\end{example}

\end{onlyenv}

\end{frame}

\begin{frame}{Squeeze theorem}


For $f\left(x\right)=x^{2}\sin x$, $-x^{2}\le f\left(x\right)\le x^{2}$,
because
\[
\lim_{x\to0}-x^{2}=\lim_{x\to0}x^{2}=0
\]
So
\[
\lim_{x\to0}x^{2}\sin x=0\,.
\]


\end{frame}

\begin{frame}{More techniques}

\begin{onlyenv}<1>

\begin{itemize}
\item For rational functions $\frac{P\left(x\right)}{Q\left(x\right)}$

\begin{enumerate}
\item Find out the highest leading term among $P\left(x\right)$ and $Q\left(x\right)$;
\item Divide $P\left(x\right)$ and $Q\left(x\right)$ by the leading term;
\item Evaluate the limit.
\end{enumerate}
\end{itemize}
\begin{example}
\[
\lim_{x\to\infty}\frac{3-x}{4+x+x^{2}}
\]
\[
\lim_{x\to-\infty}\frac{4+x^{2}-3x^{3}}{x+7x^{3}}
\]



\end{example}

\end{onlyenv}



\begin{onlyenv}<2>

\begin{itemize}
\item For $\frac{\sin mx}{nx}$

\begin{enumerate}
\item Transform
\[
\frac{\sin mx}{nx}\longrightarrow\frac{\sin mx}{mx}\cdot\frac{mx}{nx}\,;
\]

\item Use the fact to evaluate
\[
\frac{\sin mx}{mx}\,;
\]

\item Use limit laws to obtain final result.
\end{enumerate}
\end{itemize}
\end{onlyenv}



\begin{onlyenv}<3>

\begin{itemize}
\item $\left(1+mx\right)^{1/nx}$ related

\begin{enumerate}
\item Find appropriate transformation so that ``$\square$'' represent
the same thing in the expression $\left(1+\square\right)^{1/\square}$.
For example,
\[
\left(1+mx\right)^{1/nx}\longrightarrow\left[\left(1+mx\right)^{1/mx}\right]^{m/n}
\]

\item Then use the fact
\[
\lim_{\square\to0}\left(1+\square\right)^{1/\square}=e\,.
\]

\end{enumerate}
\end{itemize}
\end{onlyenv}



\begin{onlyenv}<4>

\begin{itemize}
\item Recognized as a derivative
\item l'Hospital's rule
\end{itemize}
\end{onlyenv}

\end{frame}

\begin{frame}{Continuity}

\begin{onlyenv}<1>

\begin{itemize}
\item Definition
\[
\lim_{x\to a-}f\left(x\right)=\lim_{x\to a+}f\left(x\right)=f\left(a\right)
\]

\item Terms: removable, jump, infinity discontinuity
\item Properties: Extreme value theorem, Intermediate value theorem
\item \alert{All the elementary functions and their combination and composition
are continuous in their domain!}
\item Be careful about such expression
\[
\frac{x^{2}-1}{x+1}\,.
\]

\end{itemize}
\end{onlyenv}



\begin{onlyenv}<2>

\begin{example}
p118-5, 6
\end{example}

\end{onlyenv}

\end{frame}



