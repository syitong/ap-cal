\section{Functions}
\begin{frame}{Inverse function}

\begin{onlyenv}<1>

\begin{itemize}
\item A function and its inverse function reflect the same relation from
different direction. When you buy coffee in a Starbucks, you will
deal with the following function $c=f\left(n\right)$.


\begin{center}
\begin{tabular}{|c|c|c|c|c|}
\hline 
$n$ & 1 & 2 & 3 & 4\tabularnewline
\hline 
$c$ $\left(\$\right)$ & 6 & 12 & 17 & 20\tabularnewline
\hline 
\end{tabular}
\par\end{center}


Its inverse $n=f^{-1}\left(c\right)$ is just to read the table from
second row to the first row.

\item Do not switch the letters for variables, especially in a real problem.
\item \alert{Horizontal line test} helps you determine whether the function
has an inverse or not.
\end{itemize}
\end{onlyenv}



\begin{onlyenv}<2>

\begin{itemize}
\item The domain of $f$ is the range of $f^{-1}$ and the range of $f$
is the domain of $f^{-1}$.
\item $f\circ f^{-1}\left(x\right)=x$ and $f^{-1}\circ f\left(x\right)=x$.
For example,
\[
\ln\left(e^{x}\right)=e^{\ln x}=x
\]

\end{itemize}
\end{onlyenv}



\begin{onlyenv}<3>

\begin{example}
p49-7
\end{example}

\end{onlyenv}

\end{frame}

\begin{frame}{Polynomials}

\begin{itemize}
\item $f\left(x\right)=a_{n}x^{n}+a_{n-1}x^{n-1}+\cdots+c_{1}x+c_{0}$
\item The long term behavior of a polynomial is determined by its \alert{leading term}.
This is very useful in evaluating \alert{limits}.
\item How to solve $\left(x-1\right)\left(x^{2}-2\right)<0$?
\end{itemize}
\end{frame}

\begin{frame}{Rational functions}

\begin{itemize}
\item $f\left(x\right)=\frac{P\left(x\right)}{Q\left(x\right)}$, where
$P$ and $Q$ are two polynomials.
\item How to find out the asymptotes of
\[
y=\frac{2x^{2}-x}{3x^{2}+2x-1}
\]

\item How to solve
\[
\frac{x-2}{x+1}<2x
\]

\end{itemize}
\end{frame}

\begin{frame}{Trig and anti-trig}

\begin{itemize}
\item Draw the graph of $\sin$, $\cos$, $\tan$.
\item State the the domain and range of $\arcsin/\sin^{-1}$, $\arccos/\cos^{-1}$,
$\arctan/\tan^{-1}$.\end{itemize}
\begin{example}
p50-46, p64-22
\end{example}

\end{frame}

\begin{frame}{Exp and log functions}

\begin{onlyenv}<1>

\begin{itemize}
\item Sketch the graph of $y=e^{x}$, $y=e^{-x}$, $y=\ln x$.
\item Laws of exponents:
\[
a^{0}=\quad,a^{1}=\quad,a^{m}\cdot a^{n}=\quad,a^{m}/a^{n}=\quad,
\]
\[
\left(a^{m}\right)^{n}=\quad,a^{-m}=\quad.
\]

\item Laws of log:
\[
\log_{a}1=\quad,\log_{a}a=\quad,
\]
\[
\log_{a}mn=
\]
\[
\log_{a}\frac{m}{n}=
\]
\[
\log_{a}x^{m}=
\]

\end{itemize}
\end{onlyenv}



\begin{onlyenv}<2>

\begin{example}
p61-28, p62-50, p65-38
\end{example}

\end{onlyenv}

\end{frame}

\begin{frame}{Parametric curve}

\begin{onlyenv}<1>


\[
\begin{cases}
x=f\left(t\right)\\
y=g\left(t\right)
\end{cases}
\]
gives a curve as the trace of the particle $P\left(x,y\right)$.

\end{onlyenv}



\begin{onlyenv}<2>

\begin{example}
Sketch the graph given by
\[
x=1-t,\quad y=\sqrt{t}\mbox{ for }t\ge0
\]
without using graphing calculator.
\end{example}



\begin{example}
Sketch the graph given by
\[
x=4\cos t+\cos12t,\quad y=4\sin t+\sin12t\mbox{ for }0\le t\le2\pi
\]
using graphing calculator.
\end{example}

\end{onlyenv}

\end{frame}

\begin{frame}{Polar coordinates}

\begin{itemize}
\item Sketch the following curves in polar coordinates plane
\[
r=\theta
\]
\[
r=1+3\cos\theta
\]

\end{itemize}
\end{frame}
